\documentclass[letter,12pt]{article}
\usepackage[margin=2.3cm]{geometry}
\usepackage{parskip}
\begin{document}
Name: Yan, Zi \\
Course: CIS 502 \\
Assignment: HW5 \\
\line(1,0){400}
%Write your content here

\section*{Problem 1}
\paragraph*{(a)} 
The example is below ($D_i$ is the optimal solution until day $i$):

\begin{tabular}{|c|c|c|c|c|c|c|}
\hline 
Day & 1 & \underline{2} & 3 & \underline{4} & 5 & 6 \\ 
\hline 
$x$ & 70 & 71 & 65 & 86 & 77 & 66 \\ 
\hline 
$s$ & 64 & 32 & 16 & 8 & 4 & 2 \\ 
\hline 
$D_i$ & 64 & 96 & 128 & 160 & 192 & 224 \\ 
\hline 
\end{tabular} 

The optimal solution will process 224 TB data, and system will reboot on day 
2 and day 4.

\paragraph*{(b)}
We use $D_i$ to represent the amount of TB data the system can process at 
maximum till day $i$.

For any reboot of the system, the benefit cannot reveal itself on the day of
rebooting. Therefore, we need to consider it the second day. For any 
day $i$, except first day, we find the larger amount of data can be 
processed between the system was rebooted the day before and that was 
not. Then the algorithm is described:

First, we use $k$ to represent that $s_k$ TB data can be processed by the
system today. Let $D_0 = 0$ denote on day 0 no data is processed and $k 
= 0$. Then on day 1, $D_1 =\min\{x_1, s_k\}$, and then let $k = k+1$.
Afterwards, on day $i$, $D_i = \max\{\min\{x_i, s_1\} + D_{i_2}, D_{i-1} + 
 \min\{x_i, s_k\}\}$, if first part in $\max$ function is larger, we set $k$ to
 2, otherwise we increase $k$ by 1. At last, $D_n$ is the total number of 
 terabytes processed by the optimal solution.
 
 Because every $D_i$ uses a constant time to process, including $k$, and 
 there are $n+1$ elements. Therefore, the runtime of this algorithm is
 $O(n)$.

\section*{Problem 2}

\section*{Problem 3}

\section*{Problem 4}

\end{document}
