\documentclass[letter,12pt]{article}
\usepackage[margin=2.3cm]{geometry}
\usepackage{parskip}
\begin{document}
Name: Yan, Zi \\
Course: CIS 502 \\
Assignment: HW5 \\
\line(1,0){400}
%Write your content here

\section*{Problem 1}
\paragraph*{(a)} 
The example is below ($D_i$ is the optimal solution until day $i$):

\begin{tabular}{|c|c|c|c|c|c|c|}
\hline 
Day & 1 & \underline{2} & 3 & \underline{4} & 5 & 6 \\ 
\hline 
$x$ & 70 & 71 & 65 & 86 & 77 & 66 \\ 
\hline 
$s$ & 64 & 32 & 16 & 8 & 4 & 2 \\ 
\hline 
$D_i$ & 64 & 96 & 128 & 160 & 192 & 224 \\ 
\hline 
\end{tabular} 

The optimal solution will process 224 TB data, and system will reboot on day 
2 and day 4.

\paragraph*{(b)}
We use $D_i$ to represent the amount of TB data the system can process at 
maximum till day $i$.

For any reboot of the system, the benefit cannot reveal itself on the day of
rebooting. Therefore, we need to consider it the second day. For any 
day $i$, except first day, we find the larger amount of data can be 
processed between the system was rebooted the day before and that was 
not. Then the algorithm is described:

First, we use $k$ to represent that $s_k$ TB data can be processed by the
system today. Let $D_0 = 0$ denote on day 0 no data is processed and $k 
= 0$. Then on day 1, $D_1 =\min\{x_1, s_k\}$, and then let $k = k+1$.
Afterwards, on day $i$, $D_i = \max\{\min\{x_i, s_1\} + D_{i_2}, D_{i-1} + 
 \min\{x_i, s_k\}\}$, if first part in $\max$ function is larger, we set $k$ to
 2, otherwise we increase $k$ by 1. At last, $D_n$ is the total number of 
 terabytes processed by the optimal solution.
 
 Because every $D_i$ uses a constant time to process, including $k$, and 
 there are $n+1$ elements. Therefore, the runtime of this algorithm is
 $O(n)$.

\section*{Problem 2}
For two company $i$, $j$, the rate between them is $r_{ij}$, then for $n_i$
shares of company $i$, we can trade for $n_i \times r_{ij}$ shares of 
company $j$. Thus, for a sequence of companies $i_1, i_2, ..., i_k$, the rate
between $i_n$ and $i_o$ is $r_{i_n i_o}$, where $i, o \in [1..k]$, if we want
to trade $i_1$ for $i_k$ via $i_2, i_3, ..., i_{k-1}$, the rate between $i_1$
and $i_k$ is $\prod\limits_{j=1}^{k-1} r_{i_j i_{j+1}}$. So for an opportunity
cycle, from $i_1$ via $i_2, i_3, ..., i_{k-1}$, to $i_k$ then finally back to 
$i_1$, the product of the rate, $\prod\limits_{j=1}^{k-1} r_{i_j i_{j+1}} 
\times r_{i_k i_1}$, should be greater than 1. When we apply $\log$ function
on both side, we can get $\log(\prod\limits_{j=1}^{k-1} r_{i_j i_{j+1}} 
\times r_{i_k i_1}) > 0$, namely $\sum\limits_{j=1}^{k-1} 
\log(r_{i_j i_{j+1}}) + \log(r_{i_k i_1}) > 0$. Or we can say 
$\sum\limits_{j=1}^{k-1} (-\log(r_{i_j i_{j+1}})) + (-\log(r_{i_k i_1})) < 0$.
Therefore, we can use the algorithm for finding a negative cycle in a graph.

For a graph $G=(V,E)$, let each vertex $v_{i_j}$ denote each company $i_j$,
where $j \in [1..n]$. Each edge $e_{i_j i_k}$ represents that there is a share 
trading between company $i_j$ and $i_k$, and the weight of the edge is 
$-\log(r_{i_j i_k})$, otherwise no edge is there. Then we can apply 
Negative Cycle Algorithm in the book to find out whether there is a negative
cycle in the constructed graph. If so, we can say an opportunity cycle exists
in those companies, otherwise, we say no.

The construction of such a graph takes $O(n+m)$ times, and the Negative
Cycle Algorithm takes $O(nm)$, so the total runtime is $O(nm)$.

\section*{Problem 3}
Because anyone can only notify his direct subordinates one at a time, and
after he notified one direct subordinate, the notification by this subordinate
can be performed with other notification by other subordinates in parallel.
Let $r(T)$ denote the rounds it will take to notify all people, or say nodes,
in tree $T$ recursively, and for leaf $T$, $r(T)=0$. If two subtrees $T_1$ and 
$T_2$ are the direct subordinates of a node $x$, and $r(T_1) > r(T_2)$, it 
will be better to notify $T_1$ first. Since if we notify $T_2$ first, it will take 
$\max\{1+r(T_2), 2+r(T_1)\}=2+r(T_1)$ rounds to get everybody notified, 
which is more than $\max\{1+r(T_1), 2+r(T_2)\}=1+r(T_1)$, the rounds
necessary for former case.

Here is the algorithm. For each leaf node $T$, we let $r(T) = 0$. For each
node $v$, including the root, or say the ranking office, we first sort all its  
$n$ direct subordinates, $T_1$ to $T_n$, by $r(T_i)$ in decreasing order. 
Then, $r(v) = \max\limits_{j \in [1..n]}\{j+r(T_j)\}$. So this recurrence can 
provide the minimum number of rounds each node needs to take to notify
all people in the tree that is rooted by this node.

While output the calling sequence, we can start from the ranking officer who
is notified at round 0. Each direct subordinate $T_i$ of the ranking officer 
will be notified at round $0+i$, where $i$ is the index of sorted direct 
subordinates. For each of those direct subordinates $T_i$, its direct 
subordinates $T'_j$ will be notified at round $b+j$, where $b$ is the round 
number at which $T_i$ is notified, and $j$ is the index of sorted direct 
subordinates of $T_i$. This procedure will be performed recursively, until
we reach all the leaves. And the output is all the people with the round
numbers at which they are notified.

At each node, if it has $m$ children, or say direct subordinates, it will take
$O(m\log m)$ to sort the children and $O(m)$ time to find the 
$\max\limits_{j \in [1..m]}\{j+r(T_j)\}$. Thus, for the root, it has at most
$n-1$ children, then the runtime of the algorithm is $O(n\log n)$.

\section*{Problem 4}
We have to choose $n/2$ precincts
from $n$, and ensure the chosen precincts have more than half voters for
a party and the rest $n/2$ precincts also have more than half voters for
the same party. 

We build a matrix $M[j, p_1, v_a^1, v_a^2]$, where $j$ is the number of
precincts already considered, $p_1$ is the number of precincts in partition
1, $v_a^1$ is the number of voters voting for party A in partition 1,  
$v_a^2$ is the number of voters voting for party A in partition 2, and if
this situation is possible, we set $M[j, p_1, v_a^1, v_a^2]$ to 1, otherwise
0. For first precinct, if there are $m$ voters voting for party A, then 
$M[1, 1, m, 0]=M[1, 0, 0, m] = 1$ and others will be 0. While considering
$j$th precinct with $k$ voters voting for party A, any element 
$M[j, p_1, v_a^1, v_a^2]$ will be set to 1, if $M[j-1, p_1-1,v_a^1-k, 
v_a^2]=1$ or $M[j-1,p_1, v_a^1,v_a^2-k]=1$, others will be set to 0.
After the procedure is finished, we scan the 2D matrix $M[n,n/2]$, for any
$M[n,n/2, x, y] = 1$, if both of $x$, $y$ are greater than $mn/4$ 
(then the combination will be more than $mn/2$, namely party A will win) or 
both of them are less than $mn/4$ (then the combination will be less than 
$mn/2$, namely party B will win), we can say the set of precincts is 
susceptible to gerrymandering.

In order to build the matrix, we will need $O(n^2m^2)$ time. And for the
final scan, we need $O(m^2)$ time. So total runtime is $O(n^2m^2)$.




\end{document}
