\documentclass[letter,12pt]{article}
\usepackage[margin=2.3cm]{geometry}
\usepackage{parskip}
\usepackage{amsthm}
\begin{document}
Name: Yan, Zi \\
Course: CIS 502 \\
Assignment: HW8 \\
\line(1,0){400}
%Write your content here
\section*{Question 1}
The algorithm is for each node $v$ and its left child $v'$ and right child $v''$,
if $l_{(v,v')} \neq l_{(v, v'')}$, then adjust the lengths of two edges such that
let $l_{(v,v')} = l_{(v, v'')} = \max\{l_{(v,v')}, l_{(v, v'')}\}$. After we visit each node
by using BFS and adjust corresponding edges, the resulting binary tree has
zero skew and the total edge length is as small as possible.

Observation 1: For any subtree rooted at $v$ with all its descendent leaves, $x_1$ to 
$x_k$, after we make every path from $v$ to $x_i$, where $i \in [1..k]$, equal, there 
always is a path from $v$ to a leaf $x_m$, none of whose edges will be increased, for
the case that the total length is as small as possible,
\begin{proof}
Suppose not. Let $l_i$ denote the length of the path from $v$ to $x_i$, where $i \in 
[1..k]$. Suppose we add $\Delta_i>0$ to each path from $v$ to $x_i$, such that all 
$l_i+\Delta_i$ are equal and the total length is as small as possible. But we can find the 
smallest $\Delta_m0$ and subtract $\Delta_m$ from each $\Delta_i$, such that for 
each path from $v$ to $x_i$, $(l_i+\Delta_i-\Delta_m)$ is still equal to each other and 
get the a smaller total length. Additionally, because $\Delta_m = 0$, the path from 
$v$ to $x_m$ is untouched, therefore no edges along the path from $v$ to $x_m$ is 
increased. 
\end{proof}
\begin{proof}
Suppose not. Then there are other solutions better than the algorithm provided.
Let $v$ be any node that do not follow the above algorithm.
There are three alternate solutions that can be better, 
\begin{itemize}
    \item[1)] we let $l_{(v,v')} = l_{(v, v'')} = l' >  \max\{l_{(v,v')}, l_{(v, v'')}\}$. Then at
    node $v$, all paths from $v$ are increased, which contradicts the observation 1.
    \item[2)] we let $l_{(v,v')} > l_{(v, v'')}$ after we change the lengths of the 
    corresponding edges. Therefore, the length of each path from $v''$ to a leaf has to 
    be increased by $l_{(v,v')} - l_{(v, v'')}$, in order to keep zero skew from root to 
    leaves. Then all paths from $v''$ are increased, which contradicts the observation 1.
    \item[3)] we let $l_{(v,v')} < l_{(v, v'')}$ after we change the lengths of the 
    corresponding edges. Symmetrically, we will get a contradiction.
\end{itemize}
In sum, any alternate solution is no better than the provided one. So the provided
algorithm is optimal.
\end{proof}

The BFS takes $O(n)$ for the tree traverse, and at each node the operation is constant
time, therefore, the total runtime is $O(n)$.

\section*{Question 2}
First, we make all weights of X-edges larger than Y-edges, and use Prim's algorithm
to find a minimum spanning $T_1$ with least number of X-edges $x_{min}$. Second, 
we make all weights of X-edges smaller than Y-edges, and use Prim's algorithm to find 
a minimum spanning tree $T_2$ with largest number of X-edges $x_{max}$.

If $k < x_{min}$ or $k>x_{max}$, we can conclude that no such tree that with exactly 
$k$ edges labeled $X$ exists. If $k = x_{min} = x_{max}$, then the minimum spanning 
tree produced above is the spanning tree with exactly $k$ edges labeled $X$. For the
case that $x_{min} < k < x_{max}$, we need to find a tree with exactly $k$ edges 
labeled $X$ by swapping some edges of the minimum spanning tree, $T_1$ or $T_2$ 
produced above. The swapping method is described below.

We choose to start from the spanning tree $T_1$ with $x_{min}$ edges. First, we pick a 
edge $e \in T_2 - T_1$ and the graph $T_1 \cup \{e\}$ contains a cycle $C$. Then 
the cycle must contain a edge $e' \notin T_2$. We form a new graph $T_1' = T_1 \cup 
\{e\} -\{e'\}$ and $T_1'$ has at least one edge belongs to $T_2$. Because $T_1$ can 
have at most $n-1$ edges different from $T_2$, namely $T_1$ and $T_2$ have no
common edge, therefore, after at most $n-1$ step mentioned above, $T_1$ will 
become $T_2$, and the number of X-edges in the spanning tree will be increased from
$x_{min}$ to $x_{max}$. In the process, each time only one edge is swapped, therefore
the number of X-edges in the spanning tree will increase at most 1. As a result, we can
get a exactly $k$ X-edges spanning tree during the swapping process.

The minimum spanning tree algorithm will take $O(m\log n)$. Each swapping will
take $O(m)$, and the whole swapping will take $O(mn)$. Therefore the total runtime
will be $O(mn)$

\section*{Question 3}
\paragraph*{a)}First, the general Resource Reservation Problem is NP, because, given any allocation schema, by checking each process whether its request is satisfied, we 
can tell whether there are $k$ active processes.

Second, we use independent set problem, and if we can show Independent Set $\le_P$ 
Resource Reservation Problem, then we  can conclude that the general Resource 
Reservation Problem is NP-hard.

Given any arbitrary instance of independent set problem, where the graph is $G=(V, E)$
and the size of independent set is $k$, we will first construct an equivalent instance
of resource reservation problem. We let each node $v \in V$ denote one process, and
there is an edge $(u, v)$ if process $u$ and process $v$ both request at least one
common resource. We can find a $k$-sized independent set in $G$, if and only if  
the corresponding nodes are the active processes in resource reservation problem.
\begin{proof}
First, we prove ``$\Rightarrow$". When we have a $k$-sized independent set, we
have $k$ nodes are independent, namely $k$ processes request no common resources.
Therefore, these $k$ processes can be active.

Second, we prove ``$\Leftarrow$". When we have $k$ active processes, these 
processes will request no common resources at all. Then, in the corresponding graph
$G$, these $k$ nodes will have no edge between any two of them. Therefore, these
$k$ nodes form a $k$-sized independent set.
\end{proof}

Hereby, we have resource reservation problem is both NP and NP-hard, then we can 
conclude that resource reservation problem is NP-complete.

\paragraph*{b)}
We can simply check every pair of processes to see whether their requested resources
are disjoint, and this takes $O(n^2)$.

\paragraph*{c)}
It is a bipartite problem. One type of resources, or say people, is denoted as one set of 
nodes, and the other type, or say equipment, is denoted as the other set, and there is 
an edge $(u, v)$ when a process requests people $u$ and equipment $v$. We just need 
to find $k$ matchings.
\paragraph*{d)}
It is still an NP-complete problem. Because it is a special case of a) above, when an IS 
problem is reduce to it, there are at most two edges adjacent to a single node, and we
still need to find $k$-sized independent set.

\section*{Question 4}
We first show this problem is NP. Given a set of requests $P_1, P_2, ..., P_c$, we 
examine each pair of these requests, in $O(c^2)$, we can tell whether $k$ paths exist
shared no common nodes.

Then, we need to show it is NP-hard by proving independent set $\le_P$ path selection
problem. Given any arbitrary instance of independent set problem, where the graph
is $G' = (V', E')$ and the size of independent set is $k$, we need to construct an 
equivalent instance of path selection problem. We let each node $v_i \in V'$ denote
a path $P_i$, and there is an edge $(v_i, v_j)$ if path $P_i$ and path $P_j$ share at 
least one common node. We can find a $k$-sized independent set in $G'$ if and only
if there are $k$ paths in $G$ share no common node.

\begin{proof}
``$\Rightarrow$". We have a $k$-sized independent set, namely there are $k$ nodes
that are independent in $G'$. Therefore, the paths denoted by these nodes share no 
common node in $G$, according to the definition of edge in $G'$.

``$\Leftarrow$". We have $k$ paths in $G$ which share no common node, therefore
in $G'$, the corresponding $k$ nodes do not have any edge between any pair, namely
a $k$-sized independent set.
\end{proof}

Hereby, we have path selection problem is both NP and NP-hard, so we say path 
selection problem is NP-complete.
\end{document}
