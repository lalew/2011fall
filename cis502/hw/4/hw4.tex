\documentclass[letter,12pt]{article}
\usepackage[margin=2.3cm]{geometry}
\usepackage{parskip}
\usepackage{amsthm}

\begin{document}
Name: Yan, Zi \\
Course: CIS 502 \\
Assignment: HW4 \\
\line(1,0){400}
%Write your content here

\section*{Problem 1}
Assume the sorted array is in decreasing order, if not, use $n-i$ to access
element $i$. Let $s = 1, e = n, 
mid=\lfloor (s+e)/2 \rfloor$, then compare $A[mid]$ with $mid^2$.
If $A[mid] > mid^2$, let $s = mid +1, mid = \lfloor (s+e)/2 \rfloor$;
if $A[mid] < mid^2$, let $n = mid - 1, mid = \lfloor (s+e)/2 \rfloor$;
then continue to compare $A[mid]$ and $mid^2$ recursively.
If $A[mid] = mid^2$, we just find that wanted index, but until $s > e$,
we still cannot find $A[mid] = mid^2$, we can output "does not exist".

\begin{proof}
Every time we look at the number $A[mid]$ in the middle of the array,
there are three possibilities:
\begin{itemize}
    \item If $A[mid] > mid^2$, and because the array is in decreasing
    order, we have $A[mid-1] > A[mid] > mid^2 > (mid-1)^2$, therefore
    each element before $A[mid]$ will always be greater than the square of
    its index. Consequently, we only need to look at the right part.
    \item If $A[mid] < mid ^2$, we have $A[mid+1]<A[mid]<mid^2<
    (mid+1)^2$, then all elements after $A[mid]$ will be less than their
    indices respectively. Consequently, we only need to look at the left part.
    \item If $A[mid] = mid^2$, we just find what we need.
\end{itemize}
In all these cases, we perform one probes of the array $A$ and reduce
the problem to one of at most half the size. Thus the runtime will be
$T(n) = T(n/2) + c$, namely $T(n) = O(\log n)$.
\end{proof}
\section*{Problem 2}
The algorithm is:
\begin{itemize}
    \item[1)] divide the array into two groups $g_L, g_R$ recursively.
    \item[2)] find the largest-sum subinterval in each group.
    \item[3)] while merging, first we should find the largest-sum 
    subinterval in the merged group, starting from the middle of the 
    merged group. Starting from the last element $g_L[l]$ to the left of the   
    group $g_L$, we find a index $a$, where the sum of interval $[a, l]$ is 
    the largest one in $g_L$, only if not negative. Similarly, we can find a 
    index $b$ which makes $[r, b]$ the interval with largest non-negative 
    sum, where $g_R[r]$ is the first element of $g_R$. Finally, we will get 
    the largest-sum interval $[a,b]$ by concatenating $[a, l]$ and $[r, b]$.
    \item[4)] then we choose the largest sum subinterval from those three, 
    pushing it to the upper level of merging.
\end{itemize}
\begin{proof}
The step 1 will take constant time, and the step 2 will take $T(n/2)$ time,
so the crucial part is the step 3, the merging step. Once we prove that 
in step 3, we can get the required subinterval, then we are done.

The algorithm maintains a property that every merging will provide the
largest sum of subintervals in the merged group.

\textbf{Base case:} at the bottom of the recurrence, each group will have only one number, therefore the only number is the largest-sum subinterval.

\textbf{Inductive step:} assume at level $i$, the merging result of each 
merged group is the largest-sum subinterval in it. Then we step into level
$i+1$. While merging two groups $g_L$ and $g_R$, we know from level
$i$ that intervals $[a_L, b_L], [a_R,b_R]$ are the largest-sum subintervals
in $g_L$ and $g_R$ respectively. Besides these two intervals, we should 
also consider the intervals which cross both groups, because at level $i$, we only have the largest-sum interval from each of $g_L$ and $g_R$, not
both. 

By combining the largest-sum interval ending in the last element of
$g_L$ with another starting at the first element of $g_R$, we can get the 
largest-sum interval across two group. Suppose not. Assume the 
combined interval is $[a,b]$, then there must be another interval $[a',b']$ 
which also cross two groups has larger sum than $[a,b]$. In accordance 
with the algorithm, assuming the last element of $g_L$ is $x$, the first 
element of $g_R$ is $y$, the sum of $[a, x]$ is greater than that of 
$[a', x]$, and the sum of $[y, b]$ is greater than that of $[y, b']$, so 
the sum of $[a,b]$ is greater than $[a',b']$. It contradicts with the 
assumption. So $[a, b]$ is the interval with largest sum. And the way of
finding $[a,b]$ is simply picking out the largest sum of intervals from 
the middle element to left end and right end, which should take $O(n)$
time.

Consequently, after choosing the largest one from $[a_L,b_L], [a_R,b_R]$
and $[a,b]$, we can get the largest-sum interval at level $i+1$.
\end{proof}
The runtime is $T(n)=2T(n/2) + O(n)$, namely $T(n)=O(n\log n)$

\section*{Problem 3}
We can use convolution to solve this problem.

First, we build a vector for the coefficients. 

Let $w=(\frac{c}{(n-1)^2}, 
\frac{c}{(n-2)^2}, \ldots, \frac{c}{2^2}, \frac{c}{1^2}, 0, -\frac{c}{1^2},
-\frac{c}{2^2}, \ldots, -\frac{c}{(n-2)^2}, -\frac{c}{(n-1)^2})$ and
$b=(q_n, q_{n-1}, \ldots, q_2, q_1)$. Then we can use FFT to calculate
the convolution $S_j=\sum\limits_{(k,l):k+l=2n-1+j} w_k b_l$. And 
$F_j=S_j \times q_j$.

The FFT way of the convolution will take $O(n\log n)$, and the 
multiplication will take $O(n)$. So total runtime is $O(n\log n)$.

\section*{Problem 4}
First, let $P(n)$ denote the parent of node $n$, $C_L(n)$ denote the
left child of node $n$, and $C_R(n)$ denote the right child of node $n$.

We start from the root $r$. If $x_{C_L(r)} > x_r$ and $x_{C_R(r)} > x_r$,
then $r$ is a local minimum. If not, we choose a child that has its value 
less than $x_r$ of $r$ to continue. For the condition that both of $r$'s 
children have the less value than $r$, we choose either one. Then 
recursively repeat the check procedure same as the root one, until we 
come to a leaf. During this process, there will be two situations, first there 
is a node $n$ satisfying $x_{C_L(n)} > x_n$ and $x_{C_R(n)} > x_n$, then 
$n$ is a local minimum, second we reach a leaf $n'$, then $n'$ is a local 
minimum. 

\begin{proof}
We have already proved the root. For internal nodes, once we reach a node
$n$, we know $x_{P(n)} > x_n$, because our algorithm will only choose a 
child with smaller value than its parent. Therefore, once we can provide
$x_{C_L(n)} > x_n$ and $x_{C_R(n)} > x_n$, the node $n$ is a local 
minimum. For leafs, because the implication of our algorithm makes it 
have smaller value than its parent, and its parent is the only node jointed
by edges to it, the leaf is a local minimum. 
\end{proof}

Because we only choose one child, and we probe three nodes, itself, its
left child, and its right child, at one time, the runtime $T(n)=T(n/2)+c$,
namely $T(n)=O(\log n)$

\end{document}
