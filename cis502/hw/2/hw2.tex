\documentclass[a4paper,12pt]{article}
\usepackage[margin=2.54cm]{geometry}
\usepackage{parskip}
\usepackage{tikz}
\usepackage{amsmath}
\usepackage{amssymb}
\usepackage{algorithmic}
\usepackage{algorithm}
\usepackage{amsthm}


\begin{document}
Name: Yan, Zi \\
Course: CIS 502 \\
Assignment: HW2 \\
\line(1,0){400}
%Write your content here


\section*{Problem 1}
\paragraph*{a.)}
No.
In the figure \ref{p1a}, a minimum-bottleneck tree can be \{(c,d), (d,b),
(b,e), (a,b)\}, but the minimum spanning tree is \{(a,b), (a,c), (a,d),
(b,e)\}. 

\begin{figure}[!h]
\caption{example}
\label{p1a}
\center
\begin{tikzpicture}
  \coordinate [label=170:$a$] (A) at (0,0);
  \coordinate [label=$b$] (B) at (2.5,0.25);
  \coordinate [label=$c$] (C) at (-2.5, -0.5);
  \coordinate [label=below:$d$] (D) at (0.25, -2.5);
  \coordinate [label=$e$] (E) at (0.25, 2.5);
  
  \draw (A) -- node[above] {1} (B);
  \draw (A) -- node[above] {1} (C);
  \draw (A) -- node[left] {1} (D);
  \draw (A) -- node[left] {3} (E);
  \draw (C) -- node[below] {2} (D);
  \draw (D) -- node[right] {2} (B);
  \draw (B) -- node[above] {2} (E);
\end{tikzpicture}
\end{figure}

\paragraph*{b.)}
Yes.
Suppose not. Let $T=(V, E')$ be the minimum spanning tree of the 
graph $G$, and $T'=(V, E'')$ be the minimum-bottleneck tree of $G$,
where $T$ and $T'$ are not identical. The difference is that $e \neq e'$
and $w_e < w_{e'}$, where $e \in E''$ is the most weighted edge in $T'$,
and $e' \in E'$ is the most weighted edge in $T$, but the rest edges of
two spanning tree are the same. From the assumption, we can show
that $\sum\limits_{e \in E'} w_e > \sum\limits_{e \in E''} w_e$. But this 
contradicts the fact that $T$ is MST.

\paragraph*{c.)} We can simply use Kruskal's Algorithm to find a 
minimum spanning tree in $O(m\log n)$ time. And the MST can be
regarded as a minimum-bottleneck tree.

\paragraph*{d.)} We can combine binary search and DFS to implement
the algorithm. Briefly, we use binary search to find the minimum 
bottleneck weight, using DFS to verify whether all the edges weighted 
no greater than this weight can make the original graph connected,
namely forming a spanning tree. 

\begin{proof}
Because DFS will provide a subgraph which makes all the nodes 
connected, if the original graph is connected. And once the binary
search can give a $w_{max}$ that makes the graph with all edge weights
less than or equal to $w_{max}$ connected, the DFS-MOD will also give
a connected subgraph which can be regarded as a spanning tree. As 
the binary search goes on, a minimum $w_{max}$ can be found that
it is the minimum weight and the graph with all edge weights less than
or equal to it will still be connected. At this time, DFS-MOD will provide
the minimum-bottleneck tree.

The binary search will take $O(\log m)$ time, but DFS-MOD will only
visit half of the graph at first and half of the unvisited graph or the
visited one in the subsequent steps. Therefore, the total runtime
will be $\sum\limits_{i=1}O(\frac{m+n}{2^i}) = O(m+n)=O(n+m\log n)$
\end{proof}

\begin{algorithm}
\caption{Using DFS to find minimum-bottleneck tree}
\begin{algorithmic}
\STATE \textbf{FIND\_MBT($G$)}
\WHILE{there are still more than one vertex in $G$}
\STATE Let $w_{max}$ = the median number of all existing edge 
weights
\STATE VERIFY-WEIGHT($G, w_{max}$)
\IF{$w_{max}$ is a valid weight}
    \STATE Remove all the unvisited edges from $G$
\ELSE 
    \STATE Regard all visited edges and vertices as a single node in the
following steps
\ENDIF
\ENDWHILE
\STATE The vertices and remaining edges consist of minimum 
bottleneck tree
\end{algorithmic}
\end{algorithm}

\begin{algorithm}
\begin{algorithmic}
\STATE VERIFY-WEIGHT($G, w_{max}$)
\FOR{each vertex $u \in V$}
    \STATE visited[$u$] = \FALSE
\ENDFOR
\STATE $v$ = any vertex picked from $V$
\STATE DFS-MOD($v$, $w_{max}$)
\FOR{each vertex $u \in V$}
    \IF{visited[$u$] == \FALSE}
        \RETURN "not a valid weight"
        \ELSE \RETURN "a valid weight"
    \ENDIF
\ENDFOR
\end{algorithmic}
\end{algorithm}

\begin{algorithm}
\begin{algorithmic}
\STATE DFS-MOD($u, w_{max}$)
\STATE visited[$u$] = \TRUE
\FOR{each $v \in$Adj[$u$]}
\IF{\NOT visited[$v$] \AND $w(u,v) \le w_{max}$}
\STATE DFS-MOD($v, w_{max}$)
\ENDIF
\ENDFOR
\end{algorithmic}
\end{algorithm}

\section*{Problem 2}
\paragraph*{(a)} It is true. Suppose not. Assume MACS (minimum 
altitude connected subgraph) has a distinct edge from MST (minimum 
spanning tree), connecting two nodes $i$ and $j$ to form a 
\textit{winter-optimal} path. This means that a edge $e_{\text{MACS}}$
in MACS, which is the highest edge in the path from $i$ to $j$, is 
lower than the highest edge $e_{\text{MST}}$ in MST. Therefore,
$\sum\limits_{e \in E_{\text{MACS}}}a_e < \sum\limits_{e \in 
 E_{\text{MST}}}a_e$, which contradicts the fact of MST.

%
%
%MST (minimum 
%spanning tree) has a distinct edge $e_{\text{MST}}$ from MACS 
%(minimum altitude connected subgraph), connecting two disjoint sets of 
%vertices $S$ and $V-S$, and the rest edges are the same. There are two 
%cases here.
%\begin{itemize}
%    \item[1.] $A(e_{\text{MST}}) > A(e_{\text{MACS}})$. Therefore, 
%    $\sum\limits_{e \in E_{\text{MACS}}}a_e < \sum\limits_{e \in 
%    E_{\text{MST}}}a_e$, but this contradicts that MST has the minimum 
%    total edge weight.
%    \item[2.] $A(e_{\text{MST}}) < A(e_{\text{MACS}})$. Therefore, 
%    between town $i$ and $j$, where $i \in S$ and $j \in (V-S)$ and $i$ 
%    and $j$ are directly connected by $e_{\text{MST}}$, in the MACS, the 
%    path between them will have at least $e_{\text{MACS}}$ maximum 
%    weight. In such a condition, the MACS is not optimal, which 
%    contradicts the assumption.
%\end{itemize}
%So, the MST, with respect to the edge weights $a_e$, is a MACS.

\paragraph*{(b)}
It is true. Suppose not. Assume the MACS contains no edge from the 
MST. This means in a cycle the highest edge $e_h$, which connects
the vertices $i$ and $j$, is in the MACS. Therefore removing $e_h$ and 
connecting $i$ and $j$ with the ``longer way" will form a better 
\textit{winter-optimal} path than before. This contradicts the 
assumption.


\section*{Problem 3}
Initially we need an additional array $r$, and $r$ is identical to array 
$d$, where $r_i = d_i$, and it means that $v_i$ can still form $r_i$
edges to other vertices.

The algorithm is that from $v_1$ to $v_n$ you pick a vertex one by one,
every time a vertex $v_i$ is chosen, you pick any $r_i$ other vertices,
each of which has its subscript $j$ larger than $i$ and a non-zero $r_j$ 
value, to connect to with one edge, and then decrease the corresponding
$r_j$ by 1 each time. After you go through all the vertices, all the 
element in array $r$ should be zero, otherwise the graph $G$ will 
contain either multiple edges between the same pair of nodes or self-
loop edges, or both.

The algorithm maintains that if $G$ will not contain either multiple 
edges between the same pair of nodes or self-loop edges, at any vertex 
$v_i$, each vertex $v_j$, where $j<i$, have connected to $d_j$ vertices 
with $r_j = 0$, and $r_i \le (n-i)$.
\begin{proof}
\textbf{Base:} When picking $v_1$, it is trivial.

\textbf{Induction Step:} Assume at vertex $v_i$, every vertex $v_j$,
where $j<i$, have connected to $d_j$ vertices with $r_j = 0$, and 
$r_i \le (n-i)$. So at vertex $v_{i+1}$, according to the operation
in $i$th step, there should be more than $r_i$ vertices with non-zero
$r$ value, otherwise $v_i$ will have not enough vertices to connect to,
which leads to either self-loop edges or multiple edges between $v_i$
and any other vertices, therefore contradicts the assumption. Then, 
$v_i$ can form $r_i$ edges from itself to any $r_i$ vertices, decreasing
$r_i$ to zero. For $r_{i+1}$, it should be less than or equal to $n-i$,
otherwise, multiple edges between $v_{i+1}$ and any other vertices
or self-loop edges will be formed, which contradicts the assumption.
Consequently, at $v_{i+1}$, the property still maintains.

At $v_n$, all the other vertices have their $r$ equal to zero, and $r_i \le 
(n-n) = 0$. It means every vertex $v_i$ connects to $d_i$ different
vertices with no multiple edge between the same pair of vertices and
no self-loop edge.
\end{proof}

%Suppose not, then there will be multiple edges between the same pair of 
%nodes, or self-loop edges. First, we discuss multiple edge between the 
%same pair of nodes, which means between $v_i$ and $v_j$, where $i < 
%j$, there are at least two edges. But from the algorithm above, we only 
%form exact one edge to $v_j$ when we are at $v_i$, therefore, the other
%edges are formed when we are at $v_j$. But this contradicts with our
%algorithm. Second, we discuss self-loop edges. But from the algorithm, 
%before we reach the last vertex, no self-loop edge is formed, therefore, 
%the self-loop edges will be formed when we finish the last vertex, which
%means there will be one vertex with non-zero degree. This contradicts
%the assumption.

For each vertex $v_i$, $O(d_i)$ is used to form edges and decrease 
$r_i$, so the total runtime of the algorithm should be 
$\sum\limits_{k=1}^n O(d_k) + O(n)= O(n+m)$.
\clearpage
\section*{Problem 4}
The algorithm maintains a property that every time a vertex $v$ is 
visited, dist[$v$] will hold the current shortest distance from $s$ to
$v$, and num[$v$] indicates the number of shortest paths from $s$
to $v$ with distance dist[$v$].

%Suppose not. Assume at $i$th step, the path from $s$ to $u$ is not
%the shortest one, but there is one edge $(k', j)$ instead of $(k, j)$ will
%make the path from $s$ to $k'$ to $u$ shorter than the one from $s$
%to $k$ to $u$. Therefore, dist[$k'$] + $w_{k',j} <$ dist[$k$] + $w_{k,j}$.
%And this contradicts with lines 14-20 in the algorithm.
%
%Meanwhile, lines 17-19 ensure when there are alternately shortest paths,
%num[$v$] will take those paths into account. 

\begin{proof}
\textbf{Base:} At the beginning, only $s$ is visited, therefore, dist[$s$]
=0, num[$s$]=1, and the dists of all the other nodes are $\infty$ and
the nums of them are 0.

\textbf{Induction Step:} At $i$th step, all the visited nodes will hold 
shortest distance from $s$ and how many of them. Then, at $i+1$ step,
a vertex $v$ will be visited from a vertex $u$, which is visited at $i$ 
step, on the following two conditions:

\begin{itemize}
    \item Not visited vertices. Because a node $v$ is first visited, dist[$v$]
    calculated in Line 11 is the currently shortest distance from $s$ via 
    $u$. And $v$ will also inherit the amount of shortest paths from its
    predecessor $u$.
    \item Visited vertices. If a vertex $v$ is visited before $i+1$ step, it
    means there is at least one alternative path from $s$ via $u$ to $v$.
    If the path from $s$ via $u$ to $v$ is shorter than the previous path,
    Line 14-16 will replace the distance and amount information about
    the shortest path; if the path distance is the same, Line 17-18 will 
    increase the amount of shortest paths; otherwise, the shortest path
    information will maintain unchanged.
\end{itemize}

Finally, at vertex $t$, the shortest distance from $s$ and the number of
this kind of paths will maintain.
\end{proof}

Because the algorithm just add some constant time operations, the
runtime is still $O(n+m)$.

\begin{algorithm}
\begin{algorithmic}[1]
\STATE \textbf{BFS-SHORTEST}($s$)
\STATE Set visited[$s$] = \TRUE \;and visited[$v$] = \FALSE \;for all
other $v$
\STATE Set dist[$s$] = 0 and dist[$v$] =$\infty$ for all other $v$
\STATE Set num[$s$] = 1 and num[$v$] = 0 for all other $v$
\STATE Add $s$ to Queue $Q$
\WHILE{$Q$ is not empty}
\STATE Let $u$ = Dequeue($Q$)
\FOR{all $v$ which is adjacent to $u$}
\IF{visited[$v$] == \FALSE}
\STATE Enqueue($Q$, $v$)
\STATE dist[$v$] = dist[$u$] + $w_{u,v}$
\STATE num[$v$] = num[$u$]
\ELSE
\IF{dist[$v$] $>$ dist[$u$] + $w_{u,v}$}
\STATE dist[$v$] = dist[$u$] + $w_{u,v}$
\STATE num[$v$] = 1
\ELSIF{dist[$v$] == dist[$u$] + $w_{u,v}$}
\STATE num[$v$] = num[$v$] + 1
\ENDIF
\ENDIF
\ENDFOR
\ENDWHILE
\RETURN num[$t$]
\end{algorithmic}
\end{algorithm}
\end{document}