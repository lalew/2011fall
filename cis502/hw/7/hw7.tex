\documentclass[letter,12pt]{article}
\usepackage[margin=2.3cm]{geometry}
\usepackage{parskip}
\usepackage{amsmath, amsthm, amssymb}
\begin{document}
Name: Yan, Zi \\
Course: CIS 502 \\
Assignment: HW7 \\
\line(1,0){400}
%Write your content here
\section*{Question 1}
First of all, in order to ensure that we have enough sequence of $x$ and 
sequence of $y$ to choose from, let $xs = x^n$ and $ys = y^n$, such that both
$xs$ and $ys$ will be no shorter than $s$. 

Let $m(i, j)$ denote the feasibility of getting the first $i+j$ symbols after 
interleaving first $i$ symbols of $xs$ and first $j$ symbols of $ys$. We know 
$m(0, 0) = 1$, and we will have the formula, \\
$ m(i, j) = 
\begin{cases}
1 & \text{if } (m(i - 1, j) = 1 \text{ and } xs[i] == s[i + j]) \text{ or }  
                    (m(i, j -1) = 1 \text{ and } ys[j] == s[i + j]) \\
0 & \text{otherwise}
\end{cases}$. We will have to search all the combination of $i$ ,$j$ and 
compute the corresponding $m(i, j)$, such that for $k \in [1..n]$,$i+j=k$. 
At last, if there is at least one $m(i',j') = 1$, where $i'+j' = n$, we can say $s$
is an interleaving of $x$ and $y$.

Because the length of $s$ is $n$, the maximum value of $i$ or $j$ should be 
$n$. Therefore, we need to compute a $n\times n$ matrix of $m$. The runtime 
is $O(n^2)$.
\section*{Question 2}
Observation 1: we have to order some gas on day 1. Because the tank is empty
at the end of day 0.

Observation 2: we do not need to consider the oil cost. Because for each 
possible scheme, the total cost of oil is always $\sum\limits_{i=1}^n g_i$.

Observation 3: the amount of oil we order should be $\sum\limits_{i=a}^b g_i$,
where $a \le b$, so that we can save the most, and it should be no greater than 
$L$.
\begin{proof}
Suppose not. If we order $A$ gallon oil on day $a$, we know $A>g_a$ and 
$A= \sum\limits_{i=a}^b g_i + \Delta$, where $a \le b$. So we have to order
again on day $a+b+1$. But we can actually save $c(a+b)\Delta$ by only ordering
$\sum\limits_{i=a}^b g_i$ on day $a$. It contradicts our goal that we want to 
save as much as we can.

Because the tank size is $L$, the total amount of oil ordered should be no 
greater than $L$.
\end{proof}

Observation 4: If we order oil on day $a$ and use it up on day $b$, the total
cost of storage is $c\sum\limits_{i=a}^b (i-a)g_i$. Because for day $a$ there is
no storing fee for $g_a$, for day $a+1$, the storage cost is $c(a+1-a)g_{a+1}$.
For day $i$, the storage cost is $c(i-a)g_i$. Therefore, the total cost is the sum
of the storage cost for each day, namely $c\sum\limits_{i=a}^b (i-a)g_i$.

For this problem, we have to consider it backwards from day $n$, because if we
consider it forwards, there is no constraint on how many gallon oil we need to 
order. Let OPT($i$) denote from day $i$ to day $n$, the cost is lowest. Then we
have OPT($n$) = 0. OPT($i$) = $P+ \min\{ \sum\limits_{k=i}^{j} (k-1)g_k + 
\text{OPT}(j)\}$, where $j \ge i$ and $j$ should ensure $\sum\limits_{k=i}^j g_k 
\le L$. And we need to know OPT(1).

Because the calculation of a sum is $O(n)$, and there are $n$ OPTs, the total
runtime is $O(n^2)$.

\section*{Question 3}
\paragraph*{a)}
For a bipartite graph $G=(V,E)$, each person $p_i \in V$, each night $d_i \in 
 V$, and there is a edge $(p_i, d_j) \in E$ if $d_j \notin S_i$.
 
 If $G$ has a perfect matching, each person $p_i$ will be paired with exact one 
 night $d_j$, and no person is left alone and no night is left alone. Therefore, 
 for each matching $(p_i, d_j)$, the person $p_i$ is able to cook on night $d_j$,
 and this provides a feasible schedule.
 
 If a feasible schedule exists, for a person $p_i$ and a night $d_j$, there 
 will be a pair $(p_i, d_j)$ as the assignment making $p_i$ cook on night $d_j$,
 and each person is assigned to a different night.  Therefore, in $G$, there is 
 always an edge between a $p_i$ and a $d_j$ and each $p_i$ is connected to a 
 different $d_j$, forming a perfect matching.
 \paragraph*{b)}
If either $p_i$ or $p_j$ is able to cook on night $d_l$, we are done, because
we only need to assign the person who is able to cook on night $d_l$ and keep
the other one to cook on night $d_k$. Otherwise, we use the algorithm described
below. 
 
 We build a graph $G$ described above, and a graph $A$ representing the 
 schedule of Alanis. We remove $(p_i, d_k)$ from $A$ to get $A'$. We also get 
 the residual graph $G_f$ of $A'$ with respect to $G$. If we can find a path
 from $p_j$ to $d_l$ in $G_f$, we can augment $A'$ with that path, and the 
 result is a perfect matching, namely a feasible schedule, otherwise there will be 
 no perfect matching, and no feasible schedule, either.
 
 We need $O(n^2)$ to build the graph $G$, $A'$, and $G_f$, we need another
 $O(n^2)$ to find the augmentation path. So the total runtime is $O(n^2)$.
 
 
\section*{Question 4}
We build a graph $G=(V, E)$, where there are node $s$ as a source, node $t$ as
a sink, nodes $x_i$ representing advertiser $i$, and nodes $u_j$ representing
user $j$. 1) And there is an edge between $s$ and $x_i$ with lower bound $r_i$, 
where $i \in [1..m]$.  2)There is an edge between $x_i$ and $u_j$ with capacity 
1 if $X_i \cap U_j \neq \emptyset$, where $i \in [1..m], j \in [1..n]$. 3) And there 
is an edge between each $u_j$ and $t$ with capacity 1. 4) Only $s$ has a 
demand of $- \sum\limits_{i=1}^m r_i$ and $t$ has a demand of 
$\sum\limits_{i=1}^m r_i$.

Then, if we can find a feasible circulation in $G$, we can determine that it is 
possible to assign each user to an ad conforming to the contract, because in 
the circulation, a pair $(x_i, u_j)$ denotes that an ad $i$ will be shown to user
$j$, and the lower bound $r_i$ of edge $(s, x_i)$ guarantees that the ad $i$ will 
be shown to users at least $r_i$ times. And each edge $u_j, t$ has capacity 1,
so each user is shown at most one ad.

In the other way around, if there is a feasible assignment between ads and users,
we can construct a feasible circulation as follows. We put the edge $(x_i, u_j)$
in the circulation if ad $x_i$ is shown to user $u_j$. Because each ad $x_i$ is 
shown to at least $r_i$ users, each edge $(s, x_i)$ has at least a flow of $r_i$. 
And each user is shown at most one ad, so edge $(u_j, t)$ will have at most a 
flow of 1.

Consequently, there is a feasible circulation in the graph $G$ if and only if it is
possible to assign each ad to some users conforming to the contract.

It takes us $O(nm)$ to build the graph $G$. $O(nm)$ is needed to find a feasible
circulation, if possible. So total runtime is $O(nm)$
\end{document}
