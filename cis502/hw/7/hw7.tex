\documentclass[letter,12pt]{article}
\usepackage[margin=2.3cm]{geometry}
\usepackage{parskip}
\usepackage{amsmath, amsthm, amssymb}
\begin{document}
Name: Yan, Zi \\
Course: CIS 502 \\
Assignment: HW7 \\
\line(1,0){400}
%Write your content here
\section*{Question 1}
First of all, in order to ensure that we have enough sequence of $x$ and 
sequence of $y$ to choose from, let $xs = x^n$ and $ys = y^n$, such that both
$xs$ and $ys$ will be no shorter than $s$. 

Let $m(i, j)$ denote the feasibility of getting the first $i+j$ symbols after 
interleaving first $i$ symbols of $xs$ and first $j$ symbols of $ys$. We know 
$m(0, 0) = 1$, and we will have the formula, \\
$ m(i, j) = 
\begin{cases}
1 & \text{if } (m(i - 1, j) = 1 \text{ and } xs[i] == s[i + j]) \text{ or }  
                    (m(i, j -1) = 1 \text{ and } ys[j] == s[i + j]) \\
0 & \text{otherwise}
\end{cases}$. We will have to search all the combination of $i$ ,$j$ and 
compute the corresponding $m(i, j)$, such that for $k \in [1..n]$,$i+j=k$. 
At last, if there is at least one $m(i',j') = 1$, where $i'+j' = n$, we can say $s$
is an interleaving of $x$ and $y$.

Because the length of $s$ is $n$, the maximum value of $i$ or $j$ should be 
$n$. Therefore, we need to compute a $n\times n$ matrix of $m$. The runtime 
is $O(n^2)$.
\section*{Question 2}
Observation 1: we have to order some gas on day 1. Because the tank is empty
at the end of day 0.

Observation 2: we do not need to consider the oil cost. Because for each 
possible scheme, the total cost of oil is always $\sum\limits_{i=1}^n g_i$.

Observation 3: the amount of oil we order should be $\sum\limits_{i=a}^b g_i$,
where $a \le b$.
\begin{proof}
Suppose not. If we order $A$ gallon oil on day $a$, we know $A>g_a$ and 
$A= \sum\limits_{i=a}^b g_i + \Delta$, where $a \le b$. So we have to order
again on day $a+b+1$. But we can actually save $c(a+b)\Delta$ by only ordering
$\sum\limits_{i=a}^b g_i$ on day $a$. It contradicts our goal that we want to 
save as much as we can.
\end{proof}


\section*{Question 3}
\paragraph*{a)}
For a bipartite graph $G=(V,E)$, each person $p_i \in V$, each night $d_i \in 
 V$, and there is a edge $(p_i, d_j) \in E$ if $d_j \notin S_i$.
 
 If $G$ has a perfect matching, each person $p_i$ will be paired with exact one 
 night $d_j$, and no person is left alone and no night is left alone. Therefore, 
 for each matching $(p_i, d_j)$, the person $p_i$ is able to cook on night $d_j$,
 and this provides a feasible schedule.
 
 \paragraph*{b)}
 
\section*{Question 4}


\end{document}
