\documentclass[letter,12pt]{article}
\usepackage[margin=2.3cm]{geometry}
\usepackage{parskip}
\usepackage{amsthm}

\begin{document}
Name: Yan, Zi \\
Course: CIS 502 \\
Assignment: HW3 \\
\line(1,0){400}
%Write your content here

\section*{Problem 1}
\paragraph*{(a)}
The claim is false, which means there can exist at least one stream $i$
satisfying $b_i > rt_i$.

\begin{proof}
Suppose there are two streams ($b_1, t_1$) and ($b_2, t_2$), where
$b_1 \le rt_1$, and we just need to show it is possible that $b_2 > rt_2$.

For first stream, the constraint is satisfied. So we just need to prove
that after adding second stream which has $b_2 > rt_2$, the constraint
is still satisfied. We need to show $b_1+b_2 \le r(t_1 + t_2)$. In order
to facilitate our proof, we can replace the inequalities with some equations,
$b_1 + \delta_1 = rt_1$ and $b_2 - \delta_2 = rt_2$, where $\delta_1,
\delta_2 > 0$. Then $b_1+b_2 \le r(t_1 + t_2)$ will change to $rt_1 - 
\delta_1 + rt_2 + \delta_2 \le rt_1 + rt_2$. Therefore, we need to show
$\delta_2 - \delta_1 \le 0$. So it means if the amount of bits you send
more than $rt_2$ while sending 2nd stream does not exceed the extra 
amount that you can send while sending 1st stream, the constraint will
still be satisfied. In sum, $b_2 > rt_2$ is possible.
\end{proof}

\paragraph*{(b)}
The algorithm is that first we calculate the rate $r_i$ for each time period
$t_i$ while sending $b_i$ bits, then put the streams with $r_i < r$ before
those with $r_i \geq r$. The outcome result is the schedule.

\begin{proof}
We first split all streams into two group, one for their $r_i < r$, and the
other for their $r_i \geq r$. Assume there are $k$ streams whose rates
are below $r$, therefore first group of streams are $(b_{l_1}, t_{l_1})$ to
$(b_{l_k}, t_{l_k})$, where $l_1, ..., l_k \in \{1, ..., n\}$, and $b_{l_i} \le 
rt_{l_i}$ ($1 \le i \le k$), and second group of streams are $(b_{h_1}, 
t_{h_1})$ to $(b_{h_{n-k}}, t_{h_{n-k}})$, where $h_1, ..., h_{n-k} \in 
\{1, ..., n\}$, and $b_{h_i} > rt_{h_i}$ ($1 \le i \le (n-k)$). For two groups
of streams, the inequalities can be rewritten as $b_{l_i} + \delta_{l_i} 
= rt_{l_i}$ and $b_{h_j} - \delta_{h_j} = rt_{h_j}$, where $1 \le i \le k$,
$1 \le j \le (n-k)$. And $\sum\limits_{j=1}^{n-k} \delta_{h_j} - \sum\limits_{i=1}^k \delta_{l_i} \le 0$.

It is obvious that any schedule for the $(b_{l_i}, t_{l_i})$ streams will not
violate the constraint, so we can arrange the those $k$ streams in any
order. Then we need to take care of the $(b_{h_j}, t_{h_j})$ streams.
Because we already have $\sum\limits_{j=1}^{n-k} \delta_{h_j} - \sum\limits_{i=1}^k \delta_{l_i} \le 0$, any order of arrangement of 
the rest streams will not break it as well as the constraint.

Consequently, the algorithm will give a valid schedule.
\end{proof}

The runtime consists of calculating the $r_i$ for each stream, which
will take $O(n)$ time, and putting all $(b_{l_i}, t_{l_i})$ before any
$(b_{h_j}, t_{h_j})$, which will use $O(n)$. So the total runtime is $O(n)$.

\section*{Problem 2}
\paragraph*{(a)}
Use the Interval Scheduling algorithm to find as many disjoint intervals
as possible, where the number of intervals is $n$. But slightly modify
the algorithm by taking the finish time of each selected interval as
the \textsf{status\_check} point (actually, any time point between the
start time of latest incompatible interval and the finish time of selected
interval is OK). After running the modified algorithm, all \textsf{status\_
check} points are pointed out. 

\begin{proof}
Suppose not. There is a better way to detect by only using $n-1$ 
\textsf{status\_check}. According to the Interval Scheduling algorithm,
there can be as many as $n$ disjoint process intervals, and one 
\textsf{status\_check} cannot detect any two disjoint process intervals.
Therefore, it contradicts that $n-1$ \textsf{status\_check} can suffice 
to check all the sensitive process intervals.
\end{proof}

The runtime of this algorithm is the same as Interval Scheduling 
algorithm, namely $O(n\log n)$

\paragraph*{(b)}
The claim is true.
\begin{proof}
Because there are $k^*$ disjoint processes, and no 
\textsf{status\_check} can detect any two disjoint processes, any
number, which is less than $k^*$, of \textsf{status\_check} is sure
not to detect all the sensitive processes. So there must be a set of
$k^*$ \textsf{status\_check} for all sensitive processes.
\end{proof}

\section*{Problem 3}
The algorithm is: divide the $n$ cards into two groups, then recursively 
find the most duplicated cards in each group by dividing again and 
finding the most duplicated cards in subgroups, finally merge the result.
At the merging stage, we should choose the most duplicated cards in 
each group, search the other group to find and merge any card which is 
equivalent to the selected cards, and at last choose the group of cards 
whose amount is larger as the most duplicated cards. After merging
all the groups together, if the group of most duplicated cards has 
more than $n/2$ cards, then return "Yes", otherwise "No".

\begin{proof}
Suppose not. Assume that if we choose a group of less duplicated cards
while merging, we can still find out that if $n/2$ cards are equivalent 
to each other. At level $i$, where $0 \le i \le \log n$ and level 0 means
all cards are in one group, level $\log n$ means after $\log n$ divisions
$n$ groups of one card, we have $2^i$ groups. We will choose a 
group of less duplicated cards, which means the percentage $a_{i,k}$ of 
the cards in group $k$ is fewer than 1/2, otherwise this group of cards
will be most duplicated, if $a_{i,k} \geq 1/2$. Therefore, after any 
merging, this kind of cards should still occupy fewer than half of the 
amount in the merged group, otherwise we will choose the less 
duplicated cards. Based on the assumption, we can say that, after each
merging, the selected cards will always occupy fewer than half of the
total cards in the merged group, so it is in the group after final merging.
The result contradicts that we can find out $n/2$ cards are equivalent.
\end{proof}

We will use $O(n)$ time to scan each group to find the equivalent cards
to the selected cards in the other group at merging. So the total runtime
is $T(n) = 2T(n/2) + cn$, namely $T(n)= O(n\log n)$

\section*{Problem 4}
\end{document}
