\documentclass[letter,12pt]{article}
\usepackage[margin=2.3cm]{geometry}
\usepackage{parskip}
\usepackage{amsthm}
\begin{document}
Name: Yan, Zi \\
Course: CIS 502 \\
Assignment: HW6 \\
\line(1,0){400}
%Write your content here
\section*{Question 1}
First, we use Ford-Fulkerson algorithm to compute a maximum s-t flow and 
the residual graph $G_f$. Then we use BFS to compute the corresponding 
minimum s-t cut $(A, B)$. At last, we pick arbitrary $k$ edges of set $F$ in 
the residual graph that go out of $A$. Thus, $G'=(V, E - F)$ will has as small 
as possible maximum s-t flow.

\begin{proof}
If the minimum cut of $G$ is less than $k$, we will get a 0 maximum flow in
$G'$. Let $c(A, B)$ be the minimum cut of $G$, which is the sum of the 
capacities of all the edges out of $A$ and greater than $k$. After we delete 
$k$ edges from the edges out of $A$, the minimum s-t cut is reduced to 
$c(A, B) - k$, so the maximum s-t flow is at most $c(A, B)-k$.

On the other hand, there are at least $c(A, B)$ edges out of $A$ in $G$, and 
we delete $k$ of them. Therefore, there are at least $c(A, B)-k$ edges out of
$A$ after deletion. Thus, the maximum s-t flow is at least $c(A, B)-k$.

Consequently, the maximum s-t flow will be as small as $c(A, B)-k$.
\end{proof}

The runtime of Ford-Fulkerson algorithm in this question is $O(m)$, BFS
takes $O(m)$, and the deletion will take $O(m)$. So the total runtime is
$O(m)$.

\section*{Question 2}
The algorithm is: 1) split each vertex $v$, except $s$ and $t$, into two, 
$v_{in}$ and $v_{out}$, where all the edges going into $v$ will go into 
$v_{in}$ instead, and all the edges going out of $v$ will go out of $v_{out}$ 
instead. And we connect $v_{in}$ and $v_{out}$ with a edge whose capacity 
is $c_v$, while all the nodes $v_{in}$ and $v_{out}$ will not have capacities 
any more; 2) for the edges in the original graph, we assume that they have 
infinite capacities; 3) we run Ford-Fulkerson algorithm on this modified 
graph to find the maximum s-t flow and use BFS to find the minimum s-t 
cut; 4) a edge $(v_{in}, v_{out})$ that cross the minimum cut give us the node 
$v$ is on the s-t cut for node-capacitated networks, and the cut across
all these nodes is the minimum s-t cut for node-capacitated networks.

An s-t cut for node-capacitated network is a deletion of a set of nodes,
which disconnect all the paths from $s$ to $t$.

\begin{proof}
As we define before, all edges except the ones $(v_{in}, v_{out})$ have infinite 
capacities, therefore the cut will only cross $(v_{in}, v_{out})$ edges, 
otherwise, the minimum s-t cut will always be infinite. 

In the modified graph, the Max-Flow Min-Cut Theorem applies, where the 
maximum s-t flow is the minimum cut. While transforming from the 
modified graph to the original one, we only need to contrast each pair of 
nodes $v_{in}$, $v_{out}$ to $v$, and the maximum flow does not change,
because the node capacities are respected.
If we have a cut in the original graph, we simply split the nodes as we did
before, and the maximum flow will not change as well, because the edge
capacities are respected in the modified graph. 

So the analogue of the Max-Flow Min-Cut Theorem holds true.
\end{proof} 

The runtime of modifying the graph is $O(n)$, and Ford-Fulkerson takes 
$O(m)$. So the total runtime is $O(m)$
\section*{Question 3}
\paragraph*{a)}
A $3 \times 3$ matrix, with first row 0, 1, 0, second row 1, 1, 1, and third
row 0, 1, 0. Each row and each column have at least one 1 entry, but there
is no way of rearranging the matrix, such that all diagonal entries are 1.

\paragraph*{b)}
Observation 1: If there are $n$ 1s  in the matrix, and each entry $n_k$ is 
at $(x_k, y_k)$ with $k \in [1..n]$, where $x_k \notin \{x_1, x_2, ..., x_{k-1}, 
x_{k+1},  ..., x_n\}$, and $y_k \notin \{y_1, y_2, ..., y_{k-1}, y_{k+1}, ..., 
y_n\}$, then the matrix is rearrangeable. 
\begin{proof}
We can swap the row in which the entry $n_i$ with $y_i = 1$ resides to the 
first row, then swap the row in which the entry $n_j$ with $y_j = 2$ 
resides to the second row, and so on. In sum, we just swap the row where 
the entry $n_l$ with $y_l = r$ resides to $i$th row, thus we can get a matrix
with all diagonal entries equal to 1. And each entry has different row 
number, thus any swapping will only move one entry, affecting no other
entries. Consequently, the swapping method is valid.
\end{proof}
 
 Therefore, we just need to verify whether those $n$ 1s exists. We need to
 build a graph that has a source $s$ with $-n$ demand, a sink $t$ with
 $n$ demand, a set of $n$ nodes representing row numbers, a set of $n$ 
 nodes representing column number, and a set of nodes representing the
 entries equal to 1 in the matrix. We connect $s$ to each row nodes, 
 connect each row nodes to the entry nodes residing that row, connect each  
 entry node to a column node in which the entry resides, and connect 
 all column nodes to $t$, where all the edges connecting the nodes have
 capacity 1. If we can find a feasible circulation, then the matrix is 
 rearrangeable, otherwise not.
 
 \begin{proof}
 Because the demands in $s$ and $t$, there will be at most $n$ entries equal
 to 1 will be chosen. And due to each row and each column connect to $s$
 or $t$ with capacity 1 edges, and each entry only has one row number and
 one column number, thus each entry will have unique row number and 
 column number different from each other's. And the resulting feasible 
 circulation follows the Observation 1. And the matrix is rearrangeable.
 
 On the other way, if the matrix is rearrangeable, Observation 1 follows that
 there are $n$ entries at $(1, 1), (2, 2), ..., (n, n)$. Therefore, there are $n$
 paths, where $i$th one is $s, r_i, entry(i, i), c_i, t$, showing a feasible 
 circulation.
 \end{proof}

\section*{Question 4}
First of all, we need to compute a maximum flow of $G$, and its residual
graph $G_f$. 

1) We use BFS to find all nodes that are reachable from $s$, and let them be
a set $A$, the rest be $B = V - A$. Then, the nodes in $A$ must be 
reachable from $s$ in other minimum cuts, if any. Suppose not. Let $v \in A$, 
for another cut $(A', B')$, $v \in B'$. Because $v \in A$, there is a path 
$s$-$v$ in the residual graph, but $v \in B'$, then there is an edge $(u, v)$ 
along this path, where $u \in A', v \in B'$. This contradicts that no edge 
crosses the cut $(A', B')$.

2)We use BFS to find all nodes that will reach $t$ by reversing the direction
of each edge but keeping the flow and capacities untouched. Then all found
nodes are in set $B^*$, and the rest are in set $A^* = V - B^*$. Thus, the
nodes in $B^*$ can reach $t$ in other minimum cuts, if any. The proof is 
symmetric to the one above. Suppose not. In another cut $(A^{*'}, B^{*'})$,
$v \in A^{*'}$, but $v \in B^*$. Thus there exists an edge $(v, u)$ in $B^*$, 
where $v \in A^{*'}, u \in B^{*'}$. This contradicts that no edge crosses the 
cut $(A^{*'}, B^{*'})$.

3)After we find the set $A$ and $B^*$, if the set $N = V - A - B^*$ is not
empty, the minimum s-t cut will not be unique. Because the nodes in $N$
can be either in $A$ of some minimum cut $(A, B)$, or in $B'$ of some
other minimum cut $(A', B')$. Otherwise, if $N = \emptyset$, the minimum
cut is unique.

The runtime will be $O(m)$ of Ford-Fulkerson, plus $O(m)$ of two BFS,
namely $O(m)$. 

\end{document}
