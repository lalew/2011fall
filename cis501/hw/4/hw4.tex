\documentclass[12pt,letterpaper]{article}
\usepackage[utf8x]{inputenc}
\usepackage{ucs}
\author{Zi Yan\\PennID:14137362}
\title{CIS501 Homework 4}
\date{}
\begin{document}
\maketitle

\section*{Question 1}
\paragraph*{a}
The number of bits in the block offset is 6.
\paragraph*{b}
The number of ``sets" in the cache is 256
\paragraph*{c}
The total number of block frames in the cache is $256\times2=512$.
\paragraph*{d}
The number of index bits is 8.
\paragraph*{e}
The number of tag bits is $32-6-8=18$.

\section*{Question 2}
\paragraph*{a}
The number of bits in the block offset is $\log_2B$.
\paragraph*{b}
The number of ``sets" in the cache is $\frac{S}{B\times A}$.
\paragraph*{c}
The total number of block frames in the cache is $\frac{S}{B}$.
\paragraph*{d}
The number of index bits is $\log_2\frac{S}{B\times A}$.
\paragraph*{e}
The number of tag bits is $32-\log_2B-\log_2\frac{S}{B\times A}$.
\section*{Question 3}
\paragraph*{a}
tag = 
(address$>>$(num\_index\_bits$+$num\_offset\_bits))$\&$((1$<<$num\_tag\_bits)$-$1).
\paragraph*{b}
index =
(address$>>$(num\_offset\_bits))$\&$((1$<<$num\_index\_bits)$-$1).
\section*{Question 4}
\paragraph*{a}
When the cache size is $2^{13}=8$K bytes, the miss rate is less than 10\%. When the cache
size is $2^{15}=32$K bytes, the miss rate is less than 5\%.
\paragraph*{b}
The ratio is $\frac{0.2343}{0.1533}=1.528$.
\section*{Question 5}
\paragraph*{a}
When the cache size is $2^{12}=4$K bytes, the miss rate is less than 10\%. When the cache 
size is $2^{13}=8$K bytes, the miss rate is less than 5\%.
\paragraph*{b}
The direct-mapped cache must be at most 32KB before it equals or exceeds the performance 
of the 16KB two-way set-associative cache.
\paragraph*{c}
Because direct-mapped cache has more conflict misses than set-associative cache, the
gap is mainly caused by conflict misses, and the increasing cache sizes can reduce the
conflict misses, therefore the gap narrows as the cache size increases.
\section*{Question 6}
\paragraph*{a}
When cache size is $2^{13}=8$K bytes, two write policies generate approximately the same
amount of traffic.
\paragraph*{b}
Because at large cache sizes cache misses are reduced, and write-back of dirty blocks is also
reduced, but write-through traffic does not change, therefore write-back traffic is less than
write-through.
\paragraph*{c}
Because at small cache sizes write-back of dirty block is a lot due to the high cache misses
rate, therefore write-back traffic is more than write-through.
\section*{Question 7}
\paragraph*{a}
64-byte block has the lowest miss rate.
\paragraph*{b}
8-byte block has the lowest traffic.
\paragraph*{c}
\begin{itemize}
    \item As block size increases, each write-back has to write more bytes back to lower-level
    memory.
    \item As block size increases, the set number decreases, therefore more conflict misses
    cause more write-back.
\end{itemize}
\paragraph*{d}
Today's caches are designed to minimize miss rate, because reducing the miss rate can also
reduce the traffic, and there is enough bandwidth for current cache traffic.

\section*{Question 8}
\paragraph*{a}
The accuracy of a single-entry way predictor is 65.83\%. It was more accurate than I anticipated.
Because 1) the way is either 0 or 1, thus the accuracy should be around 50\%, 2) all cache way 
information is squeezed into one predictor, then the predictor actually does not learn and 
record information about any access, therefore the predictor works no better than simply
always predicting way 0 or way 1.
\paragraph*{b}
When the predictor size is $2^{10}=1$K bytes, the two-way set associative cache is certainly
better than a direct-mapped cache.
\paragraph*{c}
There are $32\times1024\div64=512$block frames. The number of tag bits is $64-6-8=50$. 
The overhead is $2^{10}\times1+512\times50=26624\mbox{bits}=3328\mbox{bytes}$, and it 
is $3328\div32K=10.16\%$ (including predictors and tags).
\end{document}