\documentclass[12pt,letterpaper]{article}
\usepackage[utf8x]{inputenc}
\usepackage{ucs}
\author{Zi Yan}
\date{}
\title{CIS501 Homework 2b}
\begin{document}
\maketitle

\section*{Question 1}
The yield for chips of 100 $\mbox{mm}^2$ is $\frac{1}{(1+(100\times 
0.5 \times 0.004))^2} = 0.6944$.

The yield for chips of 200 $\mbox{mm}^2$ is $\frac{1}{(1+(200\times 
0.5 \times 0.004))^2} = 0.5102$.

The yield for chips of 400 $\mbox{mm}^2$ is $\frac{1}{(1+(400\times 
0.5 \times 0.004))^2} = 0.3086$.

\section*{Question 2}

\section*{Question 3}
The area of a wafer with 300 mm in diameter is 70685.83 
$\mbox{mm}^2$.

The number of good chips from each wafer for chips of 100 
$\mbox{mm}^2$ is 490.

The number of good chips from each wafer for chips of 200 
$\mbox{mm}^2$ is 180.

The number of good chips from each wafer for chips of 400 
$\mbox{mm}^2$ is 54.

\section*{Question 4}
The cost of the die manufacturing for 100 $\mbox{mm}^2$ is $25000 
\div 490 = 51.02$ dollars.

The cost of the die manufacturing for 200 $\mbox{mm}^2$ is $25000 
\div 180 = 138.89$ dollars.

The cost of the die manufacturing for 100 $\mbox{mm}^2$ is $25000 
\div 54 = 462.96$ dollars.

\section*{Question 5}
%see q5.png
\section*{Question 6}
%see q6.png
\section*{Question 7}
%see q6.png
\section*{Question 8}
At 1125 million transistors, or say 225 $\mbox{mm}^2$, the cost per 
transistor is lowest. The lowest cost per billion transistors is 242.8 dollar.

\section*{Question 9}
%see q6.png
\section*{Question 10}
At 1500 million transistors, or say 150 $\mbox{mm}^2$, the cost per
transistor is lowest. The lowest cost per billion transistors is 116.4 dollar.

\section*{Question 11}
Because either a chip may not need that as many transistors as the lowest 
cost design to implement all necessary functions, or some 
high-transistor-dense chips may be just a symbol of showing the 
company's most advanced technique. In addition, it might be difficult to 
design a chip with the number of transistors that can achieve the lowest 
cost.

\end{document}