\documentclass[12pt,letterpaper]{article}
\usepackage[utf8x]{inputenc}
\usepackage{ucs}
\author{Zi Yan\\PennID:14137362}
\title{CIS501 Homework 3b}
\date{}
\begin{document}
\maketitle
\section{Static Branch Predictor}
\paragraph*{a}
The prediction accuracy for "always taken" is 55.39\%.
\paragraph*{b}
The prediction accuracy for "always not taken" is 44.61\%.

\section{Bimodal Accuracy vs Predictor Size}
\paragraph*{a}
The best mis-prediction rate obtainable by the bimodal predictor is 12.91\%,
when the predictor number is no less than $2^{16}$.

\paragraph*{b}
For "always taken", when the predictor number is $2^{4}$, or say the size is
$2^4 \times 2 \mathrm{bits} = 2^5 \mathrm{bits} = 4 \mathrm{bytes}$, the 
mis-prediction rate is 24.92\%, about half of "always taken".

For "always not taken", when the predictor number is $2^6$, or say the size is
$2^6 \times 2 \mathrm{bits} = 2^7 \mathrm{bits} = 16 \mathrm{bytes}$, the
mis-prediction rate is 22.09\%, about half of "always not taken".

\paragraph*{c}
When the predictor number reaches $2^{16}$, the benefit reaches its top point.

\section{Gshare Accuracy vs Predictor Size}
\paragraph*{a}
The best mis-prediction rate obtainable with 8 history bits is 9.08\%, when
the predictor number is no less than $2^{16}$.

\paragraph*{b}
When the predictor number is $2^{12}$, the gshare predictor is generally better
than he simple bimodal predictor.

\paragraph*{c}
When the predictor number is no less than $2^{12}$, the gshare one is better.
Because in gshare predictor, XOR operation of the index and the global history 
can provide more information, both the location of it and the context it resides,
about a branch.

\paragraph*{d}
When the predictor number is less than $2^{12}$, the bimodal one is better 
than the gshare one. Because the predictor gets enough conflicts below the size
of $2^12$, adding information of branch history does not help, or even worsen
the situation.

\section{Gshare History Length vs Prediction Accuracy}
\paragraph*{a}
For $2^{10}$-counter predictor, when there is no history bit, the 
mis-prediction rate is the lowest. 

For $2^16$-counter predictor, when the number of history bits is 16, the
mis-prediction rate is the lowest.

\paragraph*{b}
When picking a history length, if the total number of predictor counters is 
small, like less than $2^{12}$, the history bits do not help, so do not use then,
but if the total number of predictor counters is large, like more than $2^{12}$,
the history bits can provide useful information about a branch, so use as many
history bits as you can, i.e. the same number of index bits. 

\section{Gshare Accuracy vs Predictor Size Revisited}
\paragraph*{a}
The best mis-prediction rate for gshare-n is at size of $2^{20}$.

\paragraph*{b}
The crossover point between gshare-n and bimodal is at size of $2^{12}$, 
and it is the same crossover point of gshare-8 and bimodal.

\paragraph*{c}
One property could be the connections among the adjacent branches, or say 
the global outcomes of the branches executed ahead of a branch. Because the
history bits represents the connection.

\section{Tournament Predictor Accuracy}
\paragraph*{a}
At most of time, the tournament predictor outperforms both bimodal and 
gshare, except at the size of $2^4$ and $2^5$, where bimodal is a little bit
better.

\paragraph*{b}
The tournament predictor does improve the overall peak accuracy. Because
it chooses the better predictor between bimodal and gshare. When the branch
history does not matter that much, the location might be enough, so bimodal
is better, but when the branch history matters, gshare will be selected.

\section{Tournament Predictor Accuracy -- Same Size Comparison}
\paragraph*{a}
The impact of shrinking the table sizes causes the accuracy decreased.

\paragraph*{b}
Only when the table size is larger than $2^{11}$, the tournament-fair 
outperforms both bimodal and gshare-n.

\paragraph*{c}
If the budget is tight, or say less than 4KB, I will use bimodal predictor, if
I can use more transistors, I will choose tournament of bimodal and gshare
predictor, which would outperform other predictors, like bimodal or gshare.

\end{document}