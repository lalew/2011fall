\documentclass[12pt,letterpaper]{article}
\usepackage[utf8x]{inputenc}
\usepackage{ucs}
\usepackage{graphicx}
\author{Zi Yan}
\title{CIS501 HW1b}
\date{}
\begin{document}
\maketitle

\section{Total speedup}
%\begin {figure}[!htb]
%       \includegraphics[angle=90,width=1.2\linewidth]{q5.png}
%\end {figure}
\clearpage
\section{Relative speedup}
\begin{itemize}
    \item -O3 is 3.85045 times faster than -O0.
    \item unrolled is 1.07246 times faster than -O3.
    \item unrolled \& vectorized is 1.77025 times faster than just unrolled.
    \item ``-fopenmp" is 2.14650 times faster than unrolled \& vectorized.
\end{itemize}
\section{Static instruction counts}
\begin{itemize}
    \item -O0: 15 instructions.
    \item -O3: 7 instructions.
    \item unrolled: 42 instructions.
    \item unrolled \& vectorized: 42 instructions.
\end{itemize}

\section{Dynamic instruction counts}
\begin{itemize}
    \item -O0: $65536 \times 15 = 983040$ instructions.
    \item -O3: $65536 \times 7 = 458752$ instructions.
    \item unrolled: $8192 \times 42 = 344064$ instructions.
    \item unrolled \& vectorized: $2048 \times 42 = 86016$ instructions.
\end{itemize}

\section{Static vs. dynamic instruction counts}
The most optimized .s file (unrolled \& vectorized) is 3 times as many
static instructions as the less optimized one (-O0). However, the most
optimized file has only 0.0875 times dynamic instructions of the less
optimized one. 

Apparently optimization increases the static instructions, but it
decreases the iteration times of the whole loop, therefore, the dynamic
instructions are decreased.

\section{Performance estimate based on dynamic instruction count}
Because CPI is 1, the number of clock cycles a program runs is the very
dynamic instruction number.

\begin{itemize}
    \item -O3 is 2.14286 times faster than -O0.
    \item unrolled is 1.33333 times faster than -O3.
    \item vectorized is 4 times faster than -O3 + unrolling.
\end{itemize}

\section{Loop unrolling pro/con}
\textbf{One advantage:} The program runs faster.
\\
\textbf{One disadvantage:} The size of the program doubles.

\section{Estimated vs. actual performance}
Except the performance improvement of -O3 over -O0, all the others
are high than the actual performances.

\section{CPI calculation}
\begin{itemize}
    \item -O0: CPI $=3,000,000,000 \times 45.76 \div (983040 \times
     200000) = 0.6982$.
     \item -O3: CPI $=3,000,000,000 \times 11.88 \div (458752 \times
     200000) = 0.3884$.
     \item unrolled: CPI $=3,000,000,000 \times 11.08 \div (344064
      \times 200000) = 0.4830$.
     \item unrolled+vectorized: CPI $=3,000,000,000 \times 6.26 \div
      (86016 \times 200000) = 1.0917$.
\end{itemize}

\section{CPI and compiler optimizations}
As more optimizations are enabled, the CPI goes down then climbs
up. 

From -O0 to -O3,  the compiler uses more efficient instructions for 
array multiplication and addition, like movss, mulss, addss, and 
removes the 32-bit instructions, like movl, cltq, so that CPI is 
decreased.

From -O3 to unrolled, to unrolled+vectorized, the compiler continuously
adds more complex instructions, such that the CPI is decreased by 
those instructions, because those may take more clocks to execute.


\end{document}